\section{Einleitung}

%Auftrag
%Was wird gemacht?
%Ergebnisse prägnant zusammengefasst
%Wie ist der Bericht aufgebaut?

Im Rahmen des Moduls EMV (Elektromagnetische Verträglichkeit) der Fachhochschule Nordwestschweiz wird eine Schaltung auf EMV getestet. Um die Tests durchführen zu können, wurde von den Autoren eine Schaltung zur Strommessung mit an der Fachhochschule zur Verfügung stehenden Mitteln erstellt. Ausserdem musste eruiert werden, welche Tests für die Schaltung sinnvoll sind, worauf die Autoren auf Burst- und Surgetests kamen (siehe Kapitel \ref{sec:TestUndNorm}). Da Strommessungen oft gemittelt werden ist ein kurzer Messausfall passabel. Die Schaltung muss sich im Falle einer Störung selbstständig erholen können, da solche Schaltungen oft unzugänglich verbaut sind. \\
\todo{Ergebnisse zusammengefasst zwischenschieben}\\
Im folgenden Bericht werden die Schaltung (Kapitel \ref{sec:Schaltung}), die durchgeführten Testarten und deren Normen (Kapitel \ref{sec:TestUndNorm}) und die verwendeten Schutzelemente (Kapitel \ref{sec:Schutz}) erläutert. Ausserdem werden die Testaufbauten (Kapitel \ref{sec:Testaufbau}), die Testprotokolle (Kapitel \ref{sec:Test:Burst} und \ref{sec:Test:Surge}) und die Testauswertung (Kapitel \ref{sec:Auswertung}) implementiert.